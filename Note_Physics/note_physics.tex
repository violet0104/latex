\documentclass[12pt, a4paper, oneside]{ctexbook}
\usepackage{amsmath, amsthm, amssymb, bm, graphicx, hyperref, mathrsfs, titlesec}

\title{\Huge{\textbf{大物复习}}}
\author{violet}
\date{May 2024} 
\linespread{1.5}
\linespread{1.5}
\titleformat*{\section}{\Large\bfseries\raggedright}
\newtheorem{theorem}{定理}[section]
\newtheorem{definition}[theorem]{定义}
\newtheorem{lemma}[theorem]{引理}
\newtheorem{corollary}[theorem]{推论}
\newtheorem{example}[theorem]{例}
\newtheorem{proposition}[theorem]{命题}
% 保存原始 \section格式
\newcommand{\sectionbackup}{\titleformat*{\section}{\Large\bfseries\centering}}
% 设置 \section居左显示
\titleformat*{\section}{\Large\bfseries\raggedright}

\begin{document}

\maketitle

\pagenumbering{roman}
\setcounter{page}{1}

\newpage
\pagenumbering{Roman}
\setcounter{page}{1}
\tableofcontents
\newpage
\setcounter{page}{1}
\pagenumbering{arabic}

\chapter{质点运动学}

\newpage

\section{质点运动的描述}
\subsection{位置矢量\ 运动方程\ 位移}

1.位置矢量(位矢):确定质点P某一时刻在
坐标系里的位置的物理量称位置矢量, 简称位矢
$ \boldsymbol{r} $

$$ \boldsymbol{r} =x\boldsymbol{i}+y\boldsymbol{j}+\boldsymbol{k} $$

模长为:
$$ \left|\boldsymbol{r}\right|=\sqrt{x^2+y^2+z^2} $$
方向余弦:

$$ 
    \\cos \alpha =\frac{x}{\left|\boldsymbol{r}\right|}
    ,\\cos \beta =\frac{y}{\left|\boldsymbol{r}\right|}
    ,\\cos \gamma =\frac{z}{\left|\boldsymbol{r}\right|}
$$ 

2.运动方程:$ \boldsymbol{r} $是时间t的函数,即
$$ 
    \boldsymbol{r} =\boldsymbol{r(t)}=x(t)\boldsymbol{i} +y(t)\boldsymbol{j} +z(t)\boldsymbol{k} 
$$

\newpage

\chapter{恒定磁场}

\newpage

\section{恒定电流}

\noindent 
\subsection{电流:}
通过截面S的电荷随时间的变化率(单位时间内通过截面S的电量)
$$ I=\frac{dq}{dt} $$

SI单位:A(安培)\quad
$ 1A=10^3 mA=10^6 uA $ 
\subsubsection{2.电流密度:}
细致描述导体内各点电流分布的情况

方向:$ \overrightarrow{j} $ \quad 该点正电荷运动方向

大小:单位时间内过该点且垂直于正电荷运动方向的单位面积的电荷

$$ d\boldsymbol{I}=en\vec{v_d}\cdot d\vec{S} $$
$$ \vec{j}=en\vec{v_d} $$
$$ \boldsymbol{I}=\int_S \vec{j}\cdot d\vec{S} $$

n:电子数密度

$ \vec{v_d} $:电子漂移速度

\subsubsection{3.电流的连续性方程:}
单位时间内通过闭合曲面向外流出的电荷,等于此时间内闭合曲面内电荷的减少量

即$$ \oint \vec{j}\cdot d\vec{S}=-\frac{dQ_i}{dt} $$

\subsubsection{4.恒定电流条件:}
$$ \oint \vec{j}\cdot d\vec{S}=0 \quad \text{或} \quad -\frac{dQ_i}{dt}=0 $$

\subsubsection{5.欧姆定律的微分形式:}

电阻:$$ R=\frac{\rho dl}{dS} $$

$ dI=\frac{dU}{R} ,\quad dI=\frac{1}{\rho} \frac{dU}{dl}dS ,\quad \frac{dI}{dS}= \frac{1}{\rho} \frac{dU}{dl} =\frac{1}{\rho}E $

即:$$ \vec{j}=\frac{1}{\rho} \vec{E}=\gamma \vec{E} $$
——欧姆定律的微分形式,其中 $\gamma$ 为电导率

可知:任一点的电流密度与电场强度方向相同,大小成正比

在恒定电流的情况下,电导率均匀的导体内部没有净电荷,电荷只能分布在导体的表面处(或分界面上)。\quad (可用高斯定理证明,此处省略)

\section{电源 \ 电动势}
\subsection{电动势}
\subsubsection{}


\end{document}